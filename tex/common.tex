\section{Общая часть проекта}

\subsection{Варианты использования и сценарии}

Наиболее важная часть разработки любого большого проекта - детальное проектирование. В первую очередь необходимо определить, какие задачи должна будет выполнять наша система, как задачи будут связаны между собой, какие возможности будут доступны разным группам пользователей, как будет осуществляться взаимодействие пользователя и системы. 
При разработке проекта следует двигаться от общего к частному, т.е. сначала сформулировать основные задачи в общем виде, затем постепенно детализировать их. Основные возможности нашей системы в общем виде уже были описаны во введении. Более наглядное прелставление даёт диаграмма вариантов использования. Варианты использования --- это описание действий, которые может осуществлять система в ответ на запросы пользователей или других программных систем. Варианты использования отражают функциональность системы с точки зрения пользователя. Диаграмма вариантов использования состоит из актёров (пользователь или устройство, взаимодействующе с системой извне) и собственно действий --- вариантов использования.

\includegraphics*[width=18cm]{images/usecases.png}

Диаграмма вариантов использования всё равно даёт лишь общее представление о том, какие задачи должна выполнять система. Для более детального описания используются сценарии --- текстовые описания последовательности действий. Приведём сценарии для некоторых действий, требующих более детального описания:\\



{\bf Регистрация пользователя:}

\begin{enumerate}
  \item{Программа открывает форму для регистрации, состоящую из следующих полей:}
  \begin{itemize}
    \item{логин;}
    \item{пароль;}
    \item{подтверждение пароля;}
    \item{фамилия;}
    \item{имя;}
    \item{отчество;}
    \item{e-mail;}
  \end{itemize}
  \item{Пользователь заполняет форму и нажимает кнопку "Зарегистрироваться";}
  \item{Программа проверяет валидность данных (незанятость логина, совпадение паролей);}
  \item{Если данные введены некорректно, то программа предлагает снова ввести логин, пароль и его подтверждение, иначе пользователь добавляется в базу.}
\end{enumerate}

{\bf Поиск и просмотр услуг:}
\begin{enumerate}
  \item{Пользователь вводит четыре параметра поиска: станция отправления, станция прибытия, дата отправления и вид(ы) транспорта;}
  \item{Система выбирает из базы маршрутов те маршруты, которые удовлетворяют указанным параметрам, и предоставляет список пользователю.}
\end{enumerate}

{\bf Формирование заказа:}
\begin{enumerate}
  \item{Программа формирует список рейсов, удовлетворяющих условиям поиска, и выводит его на экран с разбиением по видам транспорта (самолёты, поезда, автобусы, пароходы) и указанием наличия свободных мест;}
  \item{Если список рейсов пуст или клиента не удовлетворяет ни один из предложенных рейсов, то пользователь вводит новые параметры поиска, иначе клиент выбирает интересующий его рейс;}
  \item{Если клиент выбрал поезд, самолет или пароход, то программа предлагает выбрать категорию места с указанием цены и количества свободных мест данной категории:}
  \begin{itemize} 
    \item{для поездов: купе нижнее, купе верхнее, плацкарт боковое верхнее, плацкарт боковое нижнее, плацкарт небоковое 		нижнее, плацкарт небоковое верхнее, сидячее, общее, люкс, СВ;}
    \item{для самолетов: эконом класс, бизнес класс;}
    \item{если автобус, то переход к пункту 7;}
  \end{itemize}
  \item{Если клиент выбирал необходимую ему категорию, то:}
\begin{itemize} 
    \item{для поездов: программа формирует список вагонов, содержащих свободные места выбранной категории;}
    \item{для самолетов или пароходов: переход к пункту 7;}
\end{itemize}
  \item{Если клиента удовлетворяет один из предложенных вагонов, то он его выбирает, иначе возращается к выбору категории места;}
  \item{Программа  раскрывает список свободных мест;}
  \item{Если клиента не удовлетворяет не одно и предоставленых мест, то он возвращается на предыдущий пункт выбора, иначе выбирает конкретное место и отправляет заказ в корзину нажатием кнопки "Отправить в корзину".}
\end{enumerate}


{\bf Добавление услуг (поставщик):}
\begin{enumerate}
  \item{Поставщик выбирает вид транспорта и вводит номер маршрута;}
  \item{Поставщик формирует маршрут (добавляет станции, для каждой станции указывает время отправления и прибытия;}
  \item{Поставщик указывает даты, в которые будет следовать указанный маршрут;}
  \item{Поставщик нажимает кнопку <<Сохранить>>. Система проверяет правильность введённых данных. Если данные введены правильно, система сохраняет маршрут в базе, иначе поставщику предлагается ввести некорректные данные заново.}
\end{enumerate}

{\bf Заключение договора с поставщиком:}
\begin{enumerate}
  \item{Поставщик оставляет заявку оператору;}
  \item{В случае подтверждения оператор высылает поставщику электронную копию договора, логин и пароль;}
  \item{В случае отказа оператор объясняет поставщику причины и условия, при которых заключение договора станет возможным.}
\end{enumerate}

\endinput
