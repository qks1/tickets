\section{Введение}

Развитие сети Интернет во многом облегчает жизнь человека во многих сферах. Каждый из нас хоть раз пользовался пассажирским транспортом --- железнодорожным, авиационным, автобусным\ldots Если раньше билет можно было купить исключительно в кассе вокзала или у кондуктора, то сейчас это можно сделать, не выходя из дома, при помощи Интернета. В своей курсовой работе мы спроектировали и реализовали систему, позволяющую решить данную задачу.
В нашей системе реализованы следующие возможности:
\begin{itemize}
\item{регистрация и авторизация;}
\item{просмотр расписания между двумя станциями или по конкретной станции;}
\item{заказ билета (с выбором даты, вида транспорта, рейса, категории и номера места);}
\item{просмотр информации о поставщиках услуг;}
\end{itemize}
Для поставщиков:
\begin{itemize}
\item{заключение и расторжение договора;}
\item{добавление и редактирование услуг (маршрутов и рейсов);}
\item{просмотр статистики заказов;}
\end{itemize}

Для реализации системы была выбрана среда {\bf Ruby on Rails}, так как это относительно новая, динамично развивающаяся среда, предоставляющая огромные возможности, но в то же время довольно лёгкая в освоении. Использующаяся в ней концепция MVC (Model-View-Controller) также идеально подходит для нашей задачи. Очень большое внимание было уделено детальному проектированию системы. С этой целью использовался язык моделирования {\bf UML} и средство UML-моделирования {\bf ArgoUML}. Также важной частью курсовой работы являлась работа в коллективе и командное взаимодействие. Для эффективного взаимодействия использовалась система контроля версий {\bf Github}.

\endinput
